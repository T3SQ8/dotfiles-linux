\documentclass[a4paper]{article}

\usepackage{graphicx}
\usepackage{parskip}
\usepackage{tcolorbox}
\usepackage[margin=1in]{geometry}
\usepackage{caption}
\usepackage{float}
\captionsetup[table]{name=Tabel}
\captionsetup[figure]{name=Figur}

\newcommand{\imgcaptn}[3][width=0.5\textwidth]{
	\begin{figure}[H]
		\caption{#3}
		\begin{center}
			\includegraphics[width=0.5\textwidth]{#2}
		\end{center}
	\end{figure}
}

\newcommand{\formel}[3][]{
	\textbf{#2:}
	\ifx #1\undefined \else
	#1:
	\fi
	#3
}

\title{Titel}
\author{Name}
\date{\today}

\begin{document}

\maketitle
\begin{center}
	\begin{tabular}{ lr }
		Kurs: & Fysik 1 \\
		Medlaboranter: & Richard Martino, \\
		& Anthony Johnson \\
		Klass: & NA20a \\
	\end{tabular}
	\includegraphics[width=\textwidth]{img.png}
\end{center}

\newpage
\tableofcontents
\newpage

\section{Syfte}
\subsection{Frågeställning}
\section{Teori}
\section{Metod}
\subsection{Material}
\section{Formler}
\subsection{Beräkningar}
\section{Resultat}
\section{Diskussion}
\subsection{Analys}
\subsection{Slutsats}

\end{document}
